%!TEX root = ../thesis.tex
%*******************************************************************************
%*********************************** First Chapter *****************************
%*******************************************************************************

\chapter{Introduction}  %Title of the First Chapter

\ifpdf
    \graphicspath{{Chapter1/Figs/Raster/}{Chapter1/Figs/PDF/}{Chapter1/Figs/}}
\else
    \graphicspath{{Chapter1/Figs/Vector/}{Chapter1/Figs/}}
\fi


%********************************** %First Section  **************************************
%\section{What is loren ipsum? Title with math \texorpdfstring{$\sigma$}{[sigma]}} %Section - 1.1 

In this chapter we outline the motivation and the purpose of this thesis. We present the research questions and hypotheses that steer this study. Then, we abridge the contributions and report the thesis structure.

\section{Motivation}
Recommender Systems are software tools aimed at easing users' decision making \cite{RSdef:2015}. These systems help users in identifying information that most likely match their interests by means of suggestions that are generated by mining their behaviour data, e.g., ratings or clicks. % like explicit preferences, e.g., ratings, or their implicit feedback, e.g., clicks. %These two types of user's data are namely called: explicit and implicit feedback. 
Applications of Recommender Systems can be mainly found in web scenarios where there is a large amount of information that easily brings the user to a situation of information overload. For instance, a user who has to buy goods on Amazon, rather than browsing the 350 million items Amazon catalogue, can be suggested with a subset of items (from the catalogue) that she can likely appreciate. 

Nowadays, in the era of Ubiquitous computing, people are constantly connected to the internet via their mobile devices and move through environments that are augmented with sensors, e.g., smart cities. Therefore, user's behaviour data are fast pace generated and give us diverse information: these data not only tell us, e.g., about a user's purchased goods on an e-commerce platform, but also describe how the user interfaces herself with the surrounding physical world. Given the different type of user's behaviour information, in this thesis we distinguish between online and offline behaviour data. 

Online user's behaviour data are essentially the source of information used as input in Recommender Systems, i.e., user's actions on the web. For instance, online data may consist of the ratings given by a user to movies on streaming platforms. These ratings express the utility that the user perceives to obtain by consuming the item. Beside, online data may also be the clicks on the skip button of a media player that can be interpreted as a sign that the user didn't complete the consumption of the media content because she is not happy with it. Similarly, user's actions performed in a physical environment, e.g., staying in a place, can be exploited as proxy of the user preference.

Indeed, offline user's behaviour data consist of records of individual's physical actions like moving between two locations where user's movements data may be recorded by the sensors on the user mobile, e.g., the GPS.
Moreover, offline data may consist of payment data, that can be recorded by the user's adopted payment system, e.g., credit card, or of data collected through sensor augmented spaces. These spaces can be, e.g., shops where the shelves are equipped with IoT sensors that records which customer picked a specific product.


Although a lot of remarkable progress has been made to provide high quality recommendations to users on the web, little has been done to support users in physical spaces. A possible motivation for this can be found in how Recommender Systems, and in general information systems, have been thought: a user interacts with a system from her computer. Hence, a user has been seen as being able to interact with the virtual world via her browser from a fixed position. So, any possible interaction with the real world was neglected.

% 

%So far, the two main players in the advancements of Recommender Systems, i.e., the academia and the industry, focused on devising technological solutions to support users while these interact online, i.e., on the web.

In this thesis, we aim at paving the bridge between the current objective of Recommender Systems, that is the support of users in the online virtual environments, with the novel scenario, made possible by the advancement in sensing solutions, that consist of supporting users interacting in the physical space (offline environments). Additionally, we seek to ways to exploit the different kind of behavioural data (online and offline) to generate interesting recommendations for users that are acting in any of the two dimensions. For instance, a tourist, can be suggested with Points of Interest (POIs) to visit by exploiting her observed online behaviour, e.g., her bookmarks about activities in Florence in TripAdvisor, while she is visiting the city of Rome.



%Millions of online data are generated and made available over the internet every minute.\footnote{\url{https://web-assets.domo.com/blog/wp-content/uploads/2018/06/18-domo-data-never-sleeps-6.png}}
%For instance, users can be offered with hundreds of latest news each time they access their news feed, they can buy products from the Amazon 350 million item catalogue\footnote{\url{https://www.bigcommerce.com/blog/amazon-statistics}} or can listen to songs on Spotify 35 million item music archive.
%Consequently, users may find the process of identification and selection of the available information to be overwhelming. In order to alleviate the information overload users may encounter, Recommender Systems (RSs) have been developed \cite{RSdef:2015}. Such systems provides suggestions to the user by proposing a set of items that most likely match her interests.
%In order to suggest meaningful items to a user a Recommender System leverages two type of information: explicit and implicit feedback. The first one, is the user assessed utility related to the consumption of an item, e.g., a rating for a movie, the latter, is derived from the observations of the user's behavioural data, e.g., how much a user listened of a song.
%How to better use the different types of user's feedback have been extensively investigated in order to support users while they interact in virtual environments, i.e., the web. Concerning to users that are interacting in the real world little can be found. In this thesis we focus on bridging the gap between the two spheres in which users are acting, namely the virtual and the physical dimension. In particular, we 
%%FR Io enfatizzerei meglio che tu usi dati collezionati sia analizzando comportamenti online che off-line per dare supporto sia online che offline. Qui dici solo che dai supporto quando agiscono off-line.
%i.e., the web, but also when they act in physical environments, i.e., a city.
%This is possible due to technological advancement in the field of sensing solutions, that brought novel possibility to capture human behavioural data in real environments, i.e., recording the offline user's behaviour. Sensed user behavioural data can then be leveraged to learn user's preferences. In this thesis we mainly focus on behavioural data acquired from sensors like GPS and IoT devices, such as, beacons.

\section{Research question and hypotheses}

In this section we present the research questions and hypotheses that motivate the research presented in this thesis. \newline

% 
\noindent\textbf{How to proper model user behaviour in sequential decision making? In particular, how to design and learn the behavioural model when the user is interacting with a physical environment? (RQ1)}


%As discussed in the previous section, 
Influencing factors of user's decision making are the context in which the user operates and the order in which items are proposed to the user. To tackle these two issues Context-Aware Recommender Systems \cite{adomavicius:2011} and sequence-mining methods \cite{palumbo:2017, mobasher:2002, jannach2017} have been devised by Recommender Systems researchers.

Motivated by the willingness to better understand users preferences in different contexts and the influence of the presentation or consumption order of items, we have to identify a suitable model that can learn the user preferences from her action observations. 

%Typically, the users' explicit feedback on the outcomes of their choices (e.g., if a consumed item is liked) is very sparse. So, leveraging the user's actions observations (implicit feedback) as a proxy of her preferences has been shown that help to determine items the user will like \cite{Hu2008,Pan2008,gurabnov:2016}. When a user acts in a physical environment, like a city, it is meaningful to use the observations of her actions to identify what is relevant (preferred) to her. For instance, user's location data can be interpreted as her preferences about the places of the environment. If the user goes from place A to B rather than to place C, this can be understood as a preference for place B. As opposite to traditional information systems, where the observed user's actions are certain, % e.g., a user skips a specific song and not another. On the contrary, observations of user-space interactions are inherently uncertain. 
%In particular, 

Deriving users' preferences from their actions observations in physical spaces is not trivial because one may encounter three different issues.
Indeed, observation data of user-space interactions are recorded by sensors and are therefore affected by measurement errors. For instance, if the strength of the GPS signal is weak, the uncertainty of the measured location may be in range of dozens of meters. As a consequence, we may confound the user's locations and then assume that she visited the wrong place. Moreover, users acting in an open environment, e.g, tourists, are not constrained (guided) by a fixed navigation scheme like the links on a web page. Therefore, they may stuck moving around the same places, e.g., popular attractions. This leads to observation data describing a suboptimal behaviour. Finally, the fact that users are not willing to share much about their personal (and especially location) data \cite{location_privacy:Poikela2014} entails a lack of available databases that contain large number of individuals' user-space interaction data.
Therefore, there is the need to identify a user behaviour learning solution that can cope with scarce user's observed behavioural data. %So far, we have experimented with an approach that ``groups'' similar users together (clusters) users and then learns, by means of IRL, a generalised user behaviour model that is specific to the group.  
This question is addressed in Chapter \ref{cha:behaviour_learning} and \ref{cha:wondervalley}. \newline

\noindent\textbf{How can we use the learnt user behaviour model to generate more effective recommendations (RQ2)?}

Most of the current RS approaches do not distinguish between user behaviour learning and the recommendation  generation process. Current RS techniques, e.g., sequence-mining \cite{mobasher:2002, jannach2017}, recommend items by identifying a user's choice pattern that is directly used to identify the set of items the user is going to consume next. These are the items used for recommendation and often these recommendations are evaluated as too obvious for the target user.
Therefore, we have to identify recommendation strategies that can be used in order to increase the user satisfaction rather than suggesting what the user is predicted to consume next.
We focus on how to design such strategies so that they can be leveraged in combination with a learnt user behavioural model to generate recommendations that are of interest for a user. This problem is tackled in Chapter \ref{cha:recommendation}. \newline

\noindent \textbf{Which are the factors that make a recommendation interesting for a user? (RQ3)} 

Connecting with the previous research question, this one emphasizes the understanding of what makes a user choosing an item. 
We conjecture that there are two levels that influence the user decisions, namely a user level and an item level.
In particular, at the user level we think that influencing factors are the user intent, i.e., what is the goal that drives the user during her decision making process. In addition, at the user level we argue that the user's knowledge about the domain in which she is making choices influences her interest for an item. In particular we think that users with different levels of knowledge assesses items differently: users with an higher knowledge are able to assess a broader set of items among the available options than those users with a lower knowledge. 

Concerning the item level, specific item characteristics are here seen as the influencing factors steering the interest for an item.
Therefore, we think that by identifying the users' intent and the set of item features in the specific domain in which the user decision making process takes place, it is crucial to gain the insight to better design Recommender Systems that are capable to generate interesting recommendations to the user. This research question is investigated in Chapter \ref{cha:recommendation}. \newline

\noindent \textbf{
Is it possible to transfer the behaviour learnt from user-sapce interaction observations collected in an environment (source) to another one (target)? How do recommendations generated by leveraging the transferred user behaviour model compare with recommendations generated by using the true observations in the source environment? (RQ4)} 

This research question strictly relates with the well known problem of cold-start, i.e., few or no information about users' preferences or consumed items.
In the case of user-space interactions in the physical world, e.g., a user that visits a city for the first time, the lack of users' specific information about her preferences is a common problem. 
Among the solutions devised to alleviate the cold-start problem, Cross-domain Recommender systems \cite{crossdomain:RSdef} have been proposed. A Cross-Domain Recommender System tries to build an encompassing user preference model by mining users' preferences about items specific to a domain, e.g., books, in order to exploit this user model in other domains, e.g., movies.
We argue that by exploiting observations of user behavioural data, collected in both online and offline environments, would help to build a generalized behaviour model that can fill the gap between the two types of environments (virtual and physical) and the related user behaviour.
For instance, online collected behavioural data about visitors preferred attractions in Florence can be used to learn a generalized user behaviour model to be employed for two objectives. At first, a new generalized behaviour model can be learnt for a new city (Rome). Then, POI-visit recommendations can be generated for users that are visiting Rome.
Thus, we have to understand if there is a common set of features that can be leveraged to: (1) describe more than one physical space; (2) learn a generalized user behaviour model in one physical space and transfer the acquired knowledge to another (new) physical space. Another aspect that we aim at inquiring is how this transfer of knowledge affects the recommendations. \newline

\section{Contribution}

Here we summarize the main contributions of this thesis. Each listed item is the product of the research work steered by the research questions discussed in the previous section. \newline

\noindent \textbf{User behaviour learning from user-space interaction observations.} Concerning the first research question (RQ1) we have proposed a novel approach that accomplishes the following goals: (1) it exploits both online and offline user behavioural data; (2) it deals with situations where users' data is scarce and there is no additional information about users apart from their past observed actions. 
As application domain of our user behaviour modelling approach we selected tourism for two main reasons: firstly, it is a natural scenario in which user interacts with the surrounding environment, e.g., visiting a place or eating in a restaurant; then, users while visiting a place are in a constant sequential decision making loop, e.g., the user plans her activities and may also have to change them on-site.
In particular, we operationalized the user behaviour modelling and learning approach in two scenarios: open and closed spaces. User-space interactions in the two aforementioned scenarios are acquired either from sensors, e.g., IoT augmented spaces, or from online platforms, e.g., social media.
Experimental results have shown that by employing our modelling and learning appraoch the learnt user behaviour generalizes the true user preferences. The proposed solution can learn even in situation of scarce behavioural data by clustering like-behaving users and learning different generalized user behaviour model (one per cluster). This is possible due to the fact that learning is performed by generalizing over a set of features describing the context and the physical space in which the users' performed their actions. \newline

\noindent \textbf{Devising and evaluating Recommendation Strategies that harness a user behaviour model.} From the knowledge acquired from the user behaviour modelling and learning we have devised recommendation strategies that leverage a generalized user behaviour model. In particular, recommendations can be generated for each user behaviour model that is learnt for a cluster of like-behaving users identified in the data. Thus, different segments of users may be supported with ad-hoc suggestions. Our main contribution is to shape a class of recommendation models that by better decoupling user behaviour learning from the generation of the recommendations, provisions to a target user recommendations that do not follow his predicted behaviour. This is an aspect that have been neglected so far: most of the current Recommender System generates recommendations reproduces the user behaviour rather than supporting her in the discovery of items that she unlikely will find without the the help of the Recommender System. This can be related to the fact that systems are evaluated and hence optimized by means of the fictional train and test split of user observed data.
Offline experiments showed that the suggestions generated by our recommendation strategies provision the user with recommendations that are less accurate, in the sense that they deviate from the observed user's actions in the test data, but let the user to identify interesting (new) items that also increase the overall utility (satisfaction) of the user.

Our next contribution is to inquiry the users' perception of the recommendations generated by the recommender models we devised. Specifically we have conducted and analysed the outcomes of a user study. The user study design considered the user known items in a set of POIs and the evaluation of recommendations generated by our recommender models and baselines. The experimental results showed that our models are able to generate suggestions that are interesting POIs that are unknown to the user. This result indicates that these models may better accomplish the main task of a recommender system (in domains like tourism). \newline

\noindent \textbf{Building an application that supports users in the identification of places in the physical space.} Our next contribution is related to provide an answer to the three research questions (RQ1, RQ2 and RQ3). With the lessons learned from the user behaviour modelling and learning as well as the design and evaluation of recommendation strategies that leverage the learnt (generalized) behaviour model, we have built a mobile application that supports users in identifying relevant places in the physical space. This is necessary in order to test the models integrated in a system accessed by users while they interacts with the surrounding space.
The app helps tourists and locals in identifying POIs in a delimited area in North Italy app has been conceived to collect both the user's explicit (e.g., opinions) and implicit feedback (e.g., location). Thus, by using this app we can analyse the relations between the online and offline behaviour of users. A user can provide (online) her feedback on multiple levels: she can bookmark POIs that she finds relevant and want to visit during her travel; she can express an opinion about a POI; she can debrief her visits to POIs (user-space interactions). The offline behaviour of the user is collected by background processes that updates the user location by either using the mobile GPS sensor or interacting with an IoT network that augments the surrounding environment.  \newline

\noindent \textbf{Transfer of users' learnt behaviour.} With regard to the research question (RQ4) we have shown that by leveraging the user behaviour learnt in one city, by exploiting users' POI-visits reconstructed from an online platform, it is possible to generate recommendations in another city. Experimental results how the recommendation performance is affected when there are no observations that can be leveraged to learn a user behaviour model in an environment (target domain) and the recommendations are generated by exploiting the generalized behaviour model learnt in another environment (source domain). This situation is of pure cold-start. Moreover, we show how the learning of a generalized user behaviour model in an environemnt (target domain) can be bootstrapped by leveraging the one learnt in another domain (source domain). \newline

\section{Structure of the thesis}
We herby outline how this thesis is structured.

In Chapter \ref{cha:related work} we present the related research. In particular, we present a general overview about human behaviour modelling and prediction. Then, we introduce the state-of-the-art Recommender Systems focusing on sequential recommendations. Afterwards, we overview Inverse Reinforcement Learning and its application to human behaviour modelling. The chapter concludes with a survey of methods to collect user-space interaction data.

Chapter \ref{cha:behaviour_learning} explains the details of our approach for user's behaviour modelling and learning. In particular, in this chapter we showcase how to model users' behaviour in the context of tourism.

The recommendation strategies that supports users in finding interesting activities they can do next in the physical environment, are the subject of Chapter \ref{cha:recommendation}.

Chapter \ref{cha:transfer learning} covers the topic of transfer learning. The chapter showcase how we can learn user visit behaviour in the cities of Florence, Pisa and Rome by identifying, at first, a set of common features that describes the visit context and city own POIs. Then the chapter unveils how user's preferences over the identified set of features, e.g., learned from tourists movements data in Rome, can be leveraged to generate recommendations in another city, e.g., Florence. 

In Chapter \ref{cha:wondervalley} we present the mobile app that have been developed to test our Recommender System approach with real users that are visiting a physical space.

Chapter \ref{cha:conclusion} summarizes our findings and presents possible research aspects that branches from the current work.

\begin{comment}

The most famous equation in the world: $E^2 = (m_0c^2)^2 + (pc)^2$, which is 
known as the \textbf{energy-mass-momentum} relation as an in-line equation.

A {\em \LaTeX{} class file}\index{\LaTeX{} class file@LaTeX class file} is a file, which holds style information for a particular \LaTeX{}.


\begin{align}
CIF: \hspace*{5mm}F_0^j(a) = \frac{1}{2\pi \iota} \oint_{\gamma} \frac{F_0^j(z)}{z - a} dz
\end{align}

\nomenclature[z-cif]{$CIF$}{Cauchy's Integral Formula}                                % first letter Z is for Acronyms 
\nomenclature[a-F]{$F$}{complex function}                                                   % first letter A is for Roman symbols
\nomenclature[g-p]{$\pi$}{ $\simeq 3.14\ldots$}                                             % first letter G is for Greek Symbols
\nomenclature[g-i]{$\iota$}{unit imaginary number $\sqrt{-1}$}                      % first letter G is for Greek Symbols
\nomenclature[g-g]{$\gamma$}{a simply closed curve on a complex plane}  % first letter G is for Greek Symbols
\nomenclature[x-i]{$\oint_\gamma$}{integration around a curve $\gamma$} % first letter X is for Other Symbols
\nomenclature[r-j]{$j$}{superscript index}                                                       % first letter R is for superscripts
\nomenclature[s-0]{$0$}{subscript index}                                                        % first letter S is for subscripts


%********************************** %Second Section  *************************************
\section{Why do we use loren ipsum?} %Section - 1.2


It is a long established fact that a reader will be distracted by the readable content of a page when looking at its layout. The point of using Lorem Ipsum is that it has a more-or-less normal distribution of letters, as opposed to using `Content here, content here', making it look like readable English. Many desktop publishing packages and web page editors now use Lorem Ipsum as their default model text, and a search for `lorem ipsum' will uncover many web sites still in their infancy. Various versions have evolved over the years, sometimes by accident, sometimes on purpose (injected humour and the like).

%********************************** % Third Section  *************************************
\section{Where does it come from?}  %Section - 1.3 
\label{section1.3}

Contrary to popular belief, Lorem Ipsum is not simply random text. It has roots in a piece of classical Latin literature from 45 BC, making it over 2000 years old. Richard McClintock, a Latin professor at Hampden-Sydney College in Virginia, looked up one of the more obscure Latin words, consectetur, from a Lorem Ipsum passage, and going through the cites of the word in classical literature, discovered the undoubtable source. Lorem Ipsum comes from sections 1.10.32 and 1.10.33 of "de Finibus Bonorum et Malorum" (The Extremes of Good and Evil) by Cicero, written in 45 BC. This book is a treatise on the theory of ethics, very popular during the Renaissance. The first line of Lorem Ipsum, "Lorem ipsum dolor sit amet..", comes from a line in section 1.10.32.

The standard chunk of Lorem Ipsum used since the 1500s is reproduced below for those interested. Sections 1.10.32 and 1.10.33 from ``de Finibus Bonorum et Malorum" by Cicero are also reproduced in their exact original form, accompanied by English versions from the 1914 translation by H. Rackham

``Lorem ipsum dolor sit amet, consectetur adipisicing elit, sed do eiusmod tempor incididunt ut labore et dolore magna aliqua. Ut enim ad minim veniam, quis nostrud exercitation ullamco laboris nisi ut aliquip ex ea commodo consequat. Duis aute irure dolor in reprehenderit in voluptate velit esse cillum dolore eu fugiat nulla pariatur. Excepteur sint occaecat cupidatat non proident, sunt in culpa qui officia deserunt mollit anim id est laborum."

Section 1.10.32 of ``de Finibus Bonorum et Malorum", written by Cicero in 45 BC: ``Sed ut perspiciatis unde omnis iste natus error sit voluptatem accusantium doloremque laudantium, totam rem aperiam, eaque ipsa quae ab illo inventore veritatis et quasi architecto beatae vitae dicta sunt explicabo. Nemo enim ipsam voluptatem quia voluptas sit aspernatur aut odit aut fugit, sed quia consequuntur magni dolores eos qui ratione voluptatem sequi nesciunt. Neque porro quisquam est, qui dolorem ipsum quia dolor sit amet, consectetur, adipisci velit, sed quia non numquam eius modi tempora incidunt ut labore et dolore magnam aliquam quaerat voluptatem. Ut enim ad minima veniam, quis nostrum exercitationem ullam corporis suscipit laboriosam, nisi ut aliquid ex ea commodi consequatur? Quis autem vel eum iure reprehenderit qui in ea voluptate velit esse quam nihil molestiae consequatur, vel illum qui dolorem eum fugiat quo voluptas nulla pariatur?"

1914 translation by H. Rackham: ``But I must explain to you how all this mistaken idea of denouncing pleasure and praising pain was born and I will give you a complete account of the system, and expound the actual teachings of the great explorer of the truth, the master-builder of human happiness. No one rejects, dislikes, or avoids pleasure itself, because it is pleasure, but because those who do not know how to pursue pleasure rationally encounter consequences that are extremely painful. Nor again is there anyone who loves or pursues or desires to obtain pain of itself, because it is pain, but because occasionally circumstances occur in which toil and pain can procure him some great pleasure. To take a trivial example, which of us ever undertakes laborious physical exercise, except to obtain some advantage from it? But who has any right to find fault with a man who chooses to enjoy a pleasure that has no annoying consequences, or one who avoids a pain that produces no resultant pleasure?"

Section 1.10.33 of ``de Finibus Bonorum et Malorum", written by Cicero in 45 BC: ``At vero eos et accusamus et iusto odio dignissimos ducimus qui blanditiis praesentium voluptatum deleniti atque corrupti quos dolores et quas molestias excepturi sint occaecati cupiditate non provident, similique sunt in culpa qui officia deserunt mollitia animi, id est laborum et dolorum fuga. Et harum quidem rerum facilis est et expedita distinctio. Nam libero tempore, cum soluta nobis est eligendi optio cumque nihil impedit quo minus id quod maxime placeat facere possimus, omnis voluptas assumenda est, omnis dolor repellendus. Temporibus autem quibusdam et aut officiis debitis aut rerum necessitatibus saepe eveniet ut et voluptates repudiandae sint et molestiae non recusandae. Itaque earum rerum hic tenetur a sapiente delectus, ut aut reiciendis voluptatibus maiores alias consequatur aut perferendis doloribus asperiores repellat."

1914 translation by H. Rackham: ``On the other hand, we denounce with righteous indignation and dislike men who are so beguiled and demoralized by the charms of pleasure of the moment, so blinded by desire, that they cannot foresee the pain and trouble that are bound to ensue; and equal blame belongs to those who fail in their duty through weakness of will, which is the same as saying through shrinking from toil and pain. These cases are perfectly simple and easy to distinguish. In a free hour, when our power of choice is untrammelled and when nothing prevents our being able to do what we like best, every pleasure is to be welcomed and every pain avoided. But in certain circumstances and owing to the claims of duty or the obligations of business it will frequently occur that pleasures have to be repudiated and annoyances accepted. The wise man therefore always holds in these matters to this principle of selection: he rejects pleasures to secure other greater pleasures, or else he endures pains to avoid worse pains."

\nomenclature[z-DEM]{DEM}{Discrete Element Method}
\nomenclature[z-FEM]{FEM}{Finite Element Method}
\nomenclature[z-PFEM]{PFEM}{Particle Finite Element Method}
\nomenclature[z-FVM]{FVM}{Finite Volume Method}
\nomenclature[z-BEM]{BEM}{Boundary Element Method}
\nomenclature[z-MPM]{MPM}{Material Point Method}
\nomenclature[z-LBM]{LBM}{Lattice Boltzmann Method}
\nomenclature[z-MRT]{MRT}{Multi-Relaxation 
Time}
\nomenclature[z-RVE]{RVE}{Representative Elemental Volume}
\nomenclature[z-GPU]{GPU}{Graphics Processing Unit}
\nomenclature[z-SH]{SH}{Savage Hutter}
\nomenclature[z-CFD]{CFD}{Computational Fluid Dynamics}
\nomenclature[z-LES]{LES}{Large Eddy Simulation}
\nomenclature[z-FLOP]{FLOP}{Floating Point Operations}
\nomenclature[z-ALU]{ALU}{Arithmetic Logic Unit}
\nomenclature[z-FPU]{FPU}{Floating Point Unit}
\nomenclature[z-SM]{SM}{Streaming Multiprocessors}
\nomenclature[z-PCI]{PCI}{Peripheral Component Interconnect}
\nomenclature[z-CK]{CK}{Carman - Kozeny}
\nomenclature[z-CD]{CD}{Contact Dynamics}
\nomenclature[z-DNS]{DNS}{Direct Numerical Simulation}
\nomenclature[z-EFG]{EFG}{Element-Free Galerkin}
\nomenclature[z-PIC]{PIC}{Particle-in-cell}
\nomenclature[z-USF]{USF}{Update Stress First}
\nomenclature[z-USL]{USL}{Update Stress Last}
\nomenclature[s-crit]{crit}{Critical state}
\nomenclature[z-DKT]{DKT}{Draft Kiss Tumble}
\nomenclature[z-PPC]{PPC}{Particles per cell}
\end{comment}

%!TEX root = ../thesis.tex
%*******************************************************************************
%****************************** Third Chapter **********************************
%*******************************************************************************
\chapter{Research questions}

% **************************** Define Graphics Path **************************
\ifpdf
    \graphicspath{{Chapter3/Figs/Raster/}{Chapter3/Figs/PDF/}{Chapter3/Figs/}}
\else
    \graphicspath{{Chapter3/Figs/Vector/}{Chapter3/Figs/}}
\fi

\begin{itemize}
	\item How to proper model user behaviour in sequential decision making? In particular, how to design and learn the behavioural model when the user is interacting with a physical environment? (\textbf{RQ1})
	
	Motivated by the willingness to better understand users preferences in different contexts as well as to understand the influence of the presentation or consumption order of items we have to identify a suitable model that can learn the user preference from her action observations. As we discussed in the previous sections, typically users provides scarce feedback about the outcome of their action (i.e., if they liked or not to perform that action). Therefore, there is the need to identify a user behaviour learning solution that can cope with scarce this type of incomplete preference data. So far, we have experimented with an approach that ``groups'' similar users together (clusters) users and then learns, by means of IRL, a generalised user behaviour model that is specific to the group.  
	So far, we experimented with data that came from Location Based Social Networks (e.g., GPS traces of check-in actions). We will soon start to evaluate the proposed approach with sensor data coming from IoT devices.
	
	\item How can we use the learnt user behaviour model to generate more effective recommendations? (\textbf{RQ2})
	
	The next motivation of this study is that most of the current RS approaches do not distinguish between user behaviour learning and the recommendation  generation process. Current RS techniques (e.g., nearest neighbour) recommend items by identifying a user's choice pattern that is directly used to identify the set items the user is going to consume next. These are the items used for recommendation and often these recommendations are evaluated as too obvious for the target user.
	Therefore, we have to identify recommendation strategies that can be used in order to increase the user satisfaction rather than suggesting what the user is predicted to consume next. Furthermore, an aspect that have to be investigated is the presentation of the recommendations. In particular, in the case of Internet of Things scenarios we have to understand how to better address recommendation notifications to users.
	For instance, a user can be annoyed if he is notified every time a sensor identifies her presence close to items that are supposed to be relevant for her. Therefore, defining procedures to address this type of problems when dealing with IoT sensors is an important aspect of our research. 
	
	\item Quali sono i fattori che rendono una raccomandazione interessante per un utente ? (\textbf{RQ3})
	
	\item Which are the factors that make a recommendation interesting for a user? (\textbf{RQ4})

	\item It is possible to transfer the users' behavioural model learnt in a source domain to a target domain (\textbf{RQ5})?
	
	\item How do compare the recommendations generated by exploiting a source domain user behavioural model with those recommendation generated by harnessing the behaviour of target users (\textbf{RQ6})?
	
	
\end{itemize}

\begin{comment}

\section{First section of the third chapter}
And now I begin my third chapter here \dots

And now to cite some more people~\citet{Rea85,Ancey1996}

\subsection{First subsection in the first section}
\dots and some more 

\subsection{Second subsection in the first section}
\dots and some more \dots

\subsubsection{First subsub section in the second subsection}
\dots and some more in the first subsub section otherwise it all looks the same
doesn't it? well we can add some text to it \dots

\subsection{Third subsection in the first section}
\dots and some more \dots

\subsubsection{First subsub section in the third subsection}
\dots and some more in the first subsub section otherwise it all looks the same
doesn't it? well we can add some text to it and some more and some more and
some more and some more and some more and some more and some more \dots

\subsubsection{Second subsub section in the third subsection}
\dots and some more in the first subsub section otherwise it all looks the same
doesn't it? well we can add some text to it \dots

\section{Second section of the third chapter}
and here I write more \dots

\section{The layout of formal tables}
This section has been modified from ``Publication quality tables in \LaTeX*''
 by Simon Fear.

The layout of a table has been established over centuries of experience and 
should only be altered in extraordinary circumstances. 

When formatting a table, remember two simple guidelines at all times:

\begin{enumerate}
  \item Never, ever use vertical rules (lines).
  \item Never use double rules.
\end{enumerate}

These guidelines may seem extreme but I have
never found a good argument in favour of breaking them. For
example, if you feel that the information in the left half of
a table is so different from that on the right that it needs
to be separated by a vertical line, then you should use two
tables instead. Not everyone follows the second guideline:

There are three further guidelines worth mentioning here as they
are generally not known outside the circle of professional
typesetters and subeditors:

\begin{enumerate}\setcounter{enumi}{2}
  \item Put the units in the column heading (not in the body of
          the table).
  \item Always precede a decimal point by a digit; thus 0.1
      {\em not} just .1.
  \item Do not use `ditto' signs or any other such convention to
      repeat a previous value. In many circumstances a blank
      will serve just as well. If it won't, then repeat the value.
\end{enumerate}

A frequently seen mistake is to use `\textbackslash begin\{center\}' \dots `\textbackslash end\{center\}' inside a figure or table environment. This center environment can cause additional vertical space. If you want to avoid that just use `\textbackslash centering'


\begin{table}
\caption{A badly formatted table}
\centering
\label{table:bad_table}
\begin{tabular}{|l|c|c|c|c|}
\hline 
& \multicolumn{2}{c}{Species I} & \multicolumn{2}{c|}{Species II} \\ 
\hline
Dental measurement  & mean & SD  & mean & SD  \\ \hline 
\hline
I1MD & 6.23 & 0.91 & 5.2  & 0.7  \\
\hline 
I1LL & 7.48 & 0.56 & 8.7  & 0.71 \\
\hline 
I2MD & 3.99 & 0.63 & 4.22 & 0.54 \\
\hline 
I2LL & 6.81 & 0.02 & 6.66 & 0.01 \\
\hline 
CMD & 13.47 & 0.09 & 10.55 & 0.05 \\
\hline 
CBL & 11.88 & 0.05 & 13.11 & 0.04\\ 
\hline 
\end{tabular}
\end{table}

\begin{table}
\caption{A nice looking table}
\centering
\label{table:nice_table}
\begin{tabular}{l c c c c}
\hline 
\multirow{2}{*}{Dental measurement} & \multicolumn{2}{c}{Species I} & \multicolumn{2}{c}{Species II} \\ 
\cline{2-5}
  & mean & SD  & mean & SD  \\ 
\hline
I1MD & 6.23 & 0.91 & 5.2  & 0.7  \\

I1LL & 7.48 & 0.56 & 8.7  & 0.71 \\

I2MD & 3.99 & 0.63 & 4.22 & 0.54 \\

I2LL & 6.81 & 0.02 & 6.66 & 0.01 \\

CMD & 13.47 & 0.09 & 10.55 & 0.05 \\

CBL & 11.88 & 0.05 & 13.11 & 0.04\\ 
\hline 
\end{tabular}
\end{table}


\begin{table}
\caption{Even better looking table using booktabs}
\centering
\label{table:good_table}
\begin{tabular}{l c c c c}
\toprule
\multirow{2}{*}{Dental measurement} & \multicolumn{2}{c}{Species I} & \multicolumn{2}{c}{Species II} \\ 
\cmidrule{2-5}
  & mean & SD  & mean & SD  \\ 
\midrule
I1MD & 6.23 & 0.91 & 5.2  & 0.7  \\

I1LL & 7.48 & 0.56 & 8.7  & 0.71 \\

I2MD & 3.99 & 0.63 & 4.22 & 0.54 \\

I2LL & 6.81 & 0.02 & 6.66 & 0.01 \\

CMD & 13.47 & 0.09 & 10.55 & 0.05 \\

CBL & 11.88 & 0.05 & 13.11 & 0.04\\ 
\bottomrule
\end{tabular}
\end{table}
	
\end{comment}